\documentclass[11pt]{article}

\title{GitHub for Mathematical Research}
\author{Luke Guatelli and Andrew Penland}
\date{\today}

\begin{document}

\maketitle

\section{Introduction to GitHub}

\subsection{Overview}

\subsection{Why Use GitHub?}

\section{How To Do Specific Things}

\subsection{Create An Account}

\subsection{Start a Project}

\subsection{Add a File}

\subsection{Invite a Collaborator} 

\subsection{Issues}

Here is GitHub's description of an \textit{issue}:~\cite{github-issues} \\

\begin{quote}
Issues are used to track todos, bugs, feature requests, and more. As issues are created, they'll appear here in a searchable and filterable list. To get started, you should \underline{create an issue.}
\end{quote} 

We'll follow their advice, creating our first issue. The first issue we will create will be a ``\texttt{todo}'', telling us to create an issue. (Thus, this issue resolves itself.) 



\subsection{Pull Requests}

\subsection{Keeping Track of Changes}

\section{Conclusion}

\end{document}